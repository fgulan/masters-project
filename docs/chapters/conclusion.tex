Tehnika očitavanja rukom pisanih slova se razvija još od polovice prošlog stoljeća i od tada je u mnogočemu evoluirala. U ovom radu, kao klasifikator pojedinog slova, korištena je konvolucijska neuronska mreža.

U radu je predstavljen cjeloviti postupak prikupljanja skupa podataka i obrade slike kako bi se ulazna slika pripremila za svrhu učenja konvolucijske neuronske mreže. Također, opisana je arhitektura korištene neuronske mreže i postupak učenja iste.

Dobivena je točnost od $96.24 \%$ na skupu za ispitivanje, što je poboljšanje preko $2 \%$ u odnosu na točnost dobivenu u radu \cite{seminar} koja je iznosila $94 \%$. U navedenom radu je korištena nešto složenija arhitektura konvolucijske neuronske mreže te sličan skup podataka za učenje koji nije bio pročišćen.

Također, u radu je prikazana i primjena navedenoga modela u vidu aplikacije, odnosno tipkovnice za \emph{Appleov iOS} sustav koja se kod niza korisnika pokazala iznimnom dobrom, intuitivnom i jednostavnom za korištenje. Osobni dojam korisnika je bila i iznimna preciznost samog sustava prepoznavanja, što je vrlo dobro uzevši u obzir da je model učen na znakovima pisanim rukom na papiru gdje je dolazilo do gubitaka informacije u samom pretprocesiranju slike.

Za budući rad bi se isti taj sustav mogao isprobati i na manjim uređajima poput pametnih satova gdje fizički nije moguće smjestiti običnu tipkovnicu na zaslonu.