Očitavanje slova je problem koji se rješava već dugi niz godina. Još od 1980-ih godina, početkom ere računala, počeli su se razvijati prvi programi koji bi zamijenili čovjeka u raznim operacijama, pa tako i u očitavanju slova. Primjena je bila razna, od očitavanja brojeva računa u bankama do očitavanja poštanskih brojeva prilikom sortiranja pisama.

Kroz ovaj rad prikazana je izgradnja jednog takvog sustava za prepoznavanje rukom pisanih slova koji bi bio unaprjeđenje istog iz rada \cite{seminar}. Sustav, kao i onaj iz prije navedenoga rada, koristi konvolucijsku neuronsku mrežu iste arhitekture no uz nešto drugačiji algoritam učenja te poboljšan skup podataka za učenje. Također, prikazana je i implementacija spomenutog sustava na uređajima sa zaslonom osjetljivim na dodir, gdje korisnik, umjesto s tipkovnicom, unosi znakove crtanjem po zaslonu uređaja.

U ostatku rada bit će opisan sam sustav i arhitektura. Zatim postupak učenja istog, te primjena u vidu prepoznavanja rukom nacrtanih znakova na uređajima sa zaslonom osjetljivim na dodir.